%%%%%%%%%%%%%%%%%
% This is an sample CV template created using altacv.cls
% (v1.6.5, 3 Nov 2022) written by LianTze Lim (liantze@gmail.com). Compiles with pdfLaTeX, XeLaTeX and LuaLaTeX.
%
%% It may be distributed and/or modified under the
%% conditions of the LaTeX Project Public License, either version 1.3
%% of this license or (at your option) any later version.
%% The latest version of this license is in
%%    http://www.latex-project.org/lppl.txt
%% and version 1.3 or later is part of all distributions of LaTeX
%% version 2003/12/01 or later.
%%%%%%%%%%%%%%%%

%% Use the "normalphoto" option if you want a normal photo instead of cropped to a circle
% \documentclass[10pt,a4paper,normalphoto]{altacv}

\documentclass[10pt,letterpaper,ragged2e,withhyper]{altacv}
%% AltaCV uses the fontawesome5 and packages.
%% See http://texdoc.net/pkg/fontawesome5 for full list of symbols.

% Change the page layout if you need to
\geometry{left=1.25cm,right=1.25cm,top=1.25cm,bottom=1.2cm,columnsep=1.2cm}

% The paracol package lets you typeset columns of text in parallel
\usepackage{paracol}

% Change the font if you want to, depending on whether
% you're using pdflatex or xelatex/lualatex
\ifxetexorluatex
  % If using xelatex or lualatex:
  \setmainfont{Roboto Slab}
  \setsansfont{Lato}
  \renewcommand{\familydefault}{\sfdefault}
\else
  % If using pdflatex:
  \usepackage[rm]{roboto}
  \usepackage[defaultsans]{lato}
  % \usepackage{sourcesanspro}
  \renewcommand{\familydefault}{\sfdefault}
\fi

% Change the colours if you want to
\definecolor{SlateGrey}{HTML}{2E2E2E}
\definecolor{LightGrey}{HTML}{666666}
\definecolor{DarkPastelRed}{HTML}{450808}
\definecolor{PastelRed}{HTML}{8F0D0D}
\definecolor{GoldenEarth}{HTML}{E7D192}
\definecolor{TealBlue}{HTML}{367588}
\colorlet{name}{black}
\colorlet{tagline}{TealBlue}
\colorlet{heading}{SlateGrey}
\colorlet{headingrule}{SlateGrey}
\colorlet{subheading}{SlateGrey}
\colorlet{accent}{SlateGrey}
\colorlet{emphasis}{TealBlue}
\colorlet{body}{LightGrey}

% Change some fonts, if necessary
\renewcommand{\namefont}{\Huge\bfseries}
\renewcommand{\personalinfofont}{\footnotesize}
\renewcommand{\cvsectionfont}{\bfseries}
\renewcommand{\cvsubsectionfont}{\large\bfseries}


% Change the bullets for itemize and rating marker
% for \cvskill if you want to
\renewcommand{\itemmarker}{{\small\textbullet}}
\renewcommand{\ratingmarker}{\faCircle}

%% Use (and optionally edit if necessary) this .tex if you
%% want to use an author-year reference style like APA(6)
%% for your publication list
\input{pubs-authoryear.cfg}

%% Use (and optionally edit if necessary) this .tex if you
%% want an originally numerical reference style like IEEE
%% for your publication list
% \input{pubs-num.cfg}

%% sample.bib contains your publications
\addbibresource{publications.bib}

\begin{document}
\name{Josh Jacobson}
\tagline{Ph.D. Candidate in Applied Statistics}
%% You can add multiple photos on the left or right
% \photoR{2.8cm}{Globe_High}
% \photoL{2.5cm}{Yacht_High,Suitcase_High}

\personalinfo{%
  % Not all of these are required!
  \email{joshj@uow.edu.au}
  % \phone{000-00-0000}
  % \mailaddress{Åddrésş, Street, 00000 Cóuntry}
  % \location{Location, COUNTRY}
  \homepage{joshhjacobson.com}
  \twitter{@joshhjacobson}
  \linkedin{joshhjacobson}
  \github{joshhjacobson}
  % \orcid{0000-0003-4418-2208}
  %% You can add your own arbitrary detail with
  %% \printinfo{symbol}{detail}[optional hyperlink prefix]
  % \printinfo{\faPaw}{Hey ho!}[https://example.com/]
  %% Or you can declare your own field with
  %% \NewInfoFiled{fieldname}{symbol}[optional hyperlink prefix] and use it:
  % \NewInfoField{gitlab}{\faGitlab}[https://gitlab.com/]
  % \gitlab{your_id}
  %%
  %% For services and platforms like Mastodon where there isn't a
  %% straightforward relation between the user ID/nickname and the hyperlink,
  %% you can use \printinfo directly e.g.
  % \printinfo{\faMastodon}{@username@instace}[https://instance.url/@username]
  %% But if you absolutely want to create new dedicated info fields for
  %% such platforms, then use \NewInfoField* with a star:
  % \NewInfoField*{mastodon}{\faMastodon}
  %% then you can use \mastodon, with TWO arguments where the 2nd argument is
  %% the full hyperlink.
  % \mastodon{@username@instance}{https://instance.url/@username}
}

\makecvheader
%% Depending on your tastes, you may want to make fonts of itemize environments slightly smaller
\AtBeginEnvironment{itemize}{\small}

%% Set the left/right column width ratio to 4:6.
\columnratio{0.6}

% Start a 2-column paracol. Both the left and right columns will automatically
% break across pages if things get too long.
\vspace{0.1em}
\begin{paracol}{2}


  \cvsection{Experience}

  \cvevent{Graduate Research Fellow \& Tutor}{University of Wollongong}{Aug 2020 -- Present}{Wollongong, NSW, Australia}
  \begin{itemize}
    \item Conduct Ph.D. research developing hierarchical spatio-temporal statistical models for environmental applications, focusing especially on the global carbon cycle
    \item Awarded competitive international fellowship to carry out independently-developed research agenda
    \item Publish research in leading journals; attend and help organize scientific conferences; aid in developing and writing scientific grants
    \item Translate technical research manuscripts into public-facing content
    \item Lecture, tutor, and grade applied statistics graduate and undergraduate courses (2 semesters)
  \end{itemize}

  \vspace{-0.75em}
  \divider

  \cvevent{Data Science Consultant}{Jupiter Intelligence}{Sep 2019 -- Jan 2023}{Boulder, CO, USA}
  \begin{itemize}
    \item Developed and helped engineer an approximate Bayesian computation algorithm for parameter estimation in the company's global climate risk product; created a 16-fold speedup due to parallelization, yielding roughly \$8,000 in monthly compute cost savings
    \item Designed, helped build, and extensively tested a proprietary Python package for statistical modeling of climate extremes; the package was adopted in operational workflows to improve efficiency and reproducibility
    \item Collaborated on scientific reports for industry partners; presented research and development work at scientific conferences
    \item Full-time position during 2022, and part-time during 2019--2021
  \end{itemize}

  \vspace{-0.75em}
  \divider

  \cvevent{Undergraduate Researcher \& Tutor}{University of Colorado}{Aug 2017 -- May 2019}{Boulder, CO, USA}
  \begin{itemize}
    \item Researched and developed visualization software for multi-objective optimization problems in water-resource management
    \item Received Active Learning Award for engagements in research, service, and two professional internships
    \item Tutored and graded applied math and computer science undergraduate courses (2 semesters)
  \end{itemize}

  \medskip
  \cvsection{Service \& Leadership}

  \cvevent{Front-of-House Volunteer}{Flame Tree Community Food Co-Op}{Mar 2023 -- Present; 2 hours/week}{Thirroul, NSW, Australia}
  \begin{itemize}
    \item Ensure a positive customer experience by providing friendly, efficient service; process deliveries into inventory and maintain stock levels
  \end{itemize}

  \vspace{-0.75em}
  \divider

  \cvevent{Radio Show Host}{Probably Novel Radio Show \& Podcast}{Jan 2019 -- May 2019; 2 hours/week}{Boulder, CO, USA}
  \begin{itemize}
    \item Highlighted research achievements of STEM undergraduates in a weekly, hour-long radio interview; organized and coordinated with guests
  \end{itemize}

  \vspace{-0.75em}
  \divider

  \cvevent{Engineering Honors Program Mentor}{University of Colorado}{Aug 2016 -- May 2018; 3 hours/week}{Boulder, CO, USA}
  \begin{itemize}
    \item Led groups of first-year engineering students in readings and discussions regarding the transition to college and beyond
  \end{itemize}

  %% Switch to the right column. This will now automatically move to the second
  %% page if the content is too long.
  \switchcolumn

  \cvsection{Education}

  \cvevent{Ph.D. in Applied Statistics}{\newline \vspace{0.25ex} University of Wollongong}{2020 -- Present}{}
  
  \vspace{-0.75em}
  \divider

  \cvevent{M.S. in Applied Mathematics}{\newline \vspace{0.25ex} University of Colorado}{2018 -- 2020}{GPA: 3.8}

  \vspace{-0.75em}
  \divider

  \cvevent{B.S. with Honors in Applied Mathematics}{\newline \vspace{0.25ex} University of Colorado}{2015 -- 2019}{GPA: 3.7}
  {\small Minors: Atmospheric Science, Computer Science}

  \medskip
  \cvsection{Skills}

  % \cvsubsection{Quantitative \& Technical Skills}
  \cvevent{Quantitative \& Technical Skills}{}{}{}
  \begin{itemize}
    \item Scientific programming in Python \& R (7~years), with focuses on:
      \begin{itemize}
        \item Geospatial data analysis (e.g., dask, numpy, pandas, scipy, sf, stars, xarray)
        \item Bayesian and spatial statistical modeling (e.g., fields, FRK, gstat, PyMC, R-INLA)
        \item Data visualization (e.g., matplotlib, ggplot2)
      \end{itemize}
    \item Contributed to several open-source projects (e.g., ArviZ, Mamba, parasol, parcoords)
    \item Proficient in CDO, Git, LaTeX, Linux, and high-performance computing environments
    % \item Exceptional quantitative and statistical skills processing, analyzing, and visualizing large, multivariate datasets
    % \item Expertise developing and analyzing hierarchical statistical models with climate applications
    \item Ability to run, analyze, and interpret global atmospheric flux-inversion models
  \end{itemize}

  \vspace{-0.75em}
  \divider

  \cvevent{Communication Skills}{}{}{}
  \begin{itemize}
    \item 5 peer-reviewed scientific articles in high-impact journals, including \textit{Remote Sensing}; outlined, drafted,
    and edited manuscripts; responded to peer-reviewed critiques as both a lead and co-author
    \item 14 scientific talks, nationally and internationally, to a range of technical and non-specialist audiences; received ``best poster'' awards from professional societies
    \item Developed publication-quality figures to craft coherent stories and visualize complex results
  \end{itemize}

  \vspace{-0.75em}
  \divider

  \cvevent{Project Management Skills}{}{}{}
  \begin{itemize}
    \item Collaboratively managed multiple research projects simultaneously across groups of co-authors in multiple time zones
    \item Self-motivated: able to set and meet internal deadlines to ensure effective dissemination of research results
    \item Experience managing and mentoring an undergraduate intern on an original research project
  \end{itemize}


  %% Yeah I didn't spend too much time making all the
  %% spacing consistent... sorry. Use \smallskip, \medskip,
  %% \bigskip, \vspace etc to make adjustments.
  % \medskip


  % \cvsection{Selected Publications}

  %% Specify your last name(s) and first name(s) as given in the .bib to automatically bold your own name in the publications list.
  %% One caveat: You need to write \bibnamedelima where there's a space in your name for this to work properly; or write \bibnamedelimi if you use initials in the .bib
  %% You can specify multiple names, especially if you have changed your name or if you need to highlight multiple authors.
  % \mynames{Jacobson/Josh}
  %% MAKE SURE THERE IS NO SPACE AFTER THE FINAL NAME IN YOUR \mynames LIST

  % \nocite{*}
  % \printbibliography[heading=none]


\end{paracol}


\end{document}
